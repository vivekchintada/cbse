\documentclass[12pt,-letter paper]{article}                       
\usepackage{siunitx}                                              
\usepackage{setspace}
\usepackage{gensymb}                                              
\usepackage{xcolor}                                               
\usepackage{caption}
%\usepackage{subcaption}
\doublespacing                                                    
\singlespacing                                                    
\usepackage[none]{hyphenat}
\usepackage{amssymb}
\usepackage{relsize}
\usepackage[cmex10]{amsmath}
\usepackage{mathtools}
\usepackage{amsmath}                                              
\usepackage{commath}                                              
\usepackage{amsthm}
\interdisplaylinepenalty=2500
%\savesymbol{iint}\usepackage{txfonts}                                              
%\restoresymbol{TXF}{iint}                                        
\usepackage{wasysym}                                              
\usepackage{amsthm}
\usepackage{mathrsfs}                                             
\usepackage{txfonts}                                              
\let\vec\mathbf{}
\usepackage{stfloats}
\usepackage{float}
\usepackage{cite}
\usepackage{cases}                                                
\usepackage{subfig}                                               
%\usepackage{xtab}
\usepackage{longtable}
\usepackage{multirow}
%\usepackage{algorithm}
\usepackage{amssymb}
%\usepackage{algpseudocode}
\usepackage{enumitem}
\usepackage{mathtools}
%\usepackage{eenrc}
%\usepackage[framemethod=tikz]{mdframed}                          
\usepackage{listings}                                             
%\usepackage{listings}
\usepackage[latin1]{inputenc}
%%\usepackage{color}{
%%\usepackage{lscape}
\usepackage{textcomp}
\usepackage{titling}
\usepackage{hyperref}
%\usepackage{fulbigskip}
\usepackage{tikz}
\usepackage{graphicx}                                             
\lstset{
  frame=single,
  breaklines=true
}
\let\vec\mathbf{}
\usepackage{enumitem}                                             
\usepackage{graphicx}                                             
\usepackage{siunitx}
\let\vec\mathbf{}                                                 
\usepackage{enumitem}
\usepackage{graphicx}
\usepackage{enumitem}
\usepackage{tfrupee}
\usepackage{amsmath}
\usepackage{amssymb}
\usepackage{mwe} % for blindtext and example-image-a in example
\usepackage{wrapfig}


\begin{document}


	\begin{enumerate}	
\item  Find the angle between the line $\vec{r} = (2\vec{i}-\vec{j}+\vec{k})+ \lambda(3\vec{i}-\vec{j}+2\vec{k}$ and the plane $\vec{r}\cdot(\vec{i} +\vec{j}+\vec{k}) =3$.

\item Find the co-ordinates of the point, where the line $\frac{x+2}{1}=\frac{y-5}{3}=\frac{z+1}{5}$ cuts the $yz$-plane.

\item if \begin{align*}y= 5e^{7x}+6e^{-7x}\end{align*}  , show that $\frac{d^2y}{dx^2}=49y$.

\item if $A$ is a square matrix of order $2$ and $\begin{vmatrix}
		A
		\end{vmatrix} = 4$, then find the value of $\begin{vmatrix} 2AA^\prime\end{vmatrix}$, where $A^\prime$ is the transpose of matrix A.

\item Find the order of differential equation of the family of circles of radius $3$ units.

\item Find the value of $(x-y)$ from the matrix equation
	$
	2\begin{bmatrix}
	x & 5 \\
	7 & y-3 \\
	\end{bmatrix}+
	\begin{bmatrix}
	    -3 & -4 \\
	    1  & 2 \\
	\end{bmatrix}=
	\begin{bmatrix}
		7 & 6 \\
		15 & 14 \\
	\end{bmatrix}
        $
\item solve the equation differential equation \begin{align*}(y+3x^2)\frac{dx}{dy}=x\end{align*}.

\item Find \begin{align*}\int e^x(\cos x-\sin x)\csc^2 x dx\end{align*}

\item Using vectors, prove that the points $(2,-1,3), (3,-5,1)$ and $(-1,11,9)$ are collinear.

\item For any two vectors $\vec{a}$ and $\vec{b}$ , prove that \begin{align*}(\vec{a}*\vec{b})^2=\vec{a}^2\vec{b}^2-(\vec{a}.\vec{b})^2\end{align*}

\item Find \begin{align*}\int \frac{x-1}{(x-2)(x-3)}\,dx\end{align*}

	\item Integrate \begin{align*}\frac{e^x}{\sqrt{5-4e^x-e^{2x}}}\end{align*} with respect to $x$.

\item If $P(A)=0.6, P(B)=0.5$ and $P(B|A)=0.4$, find $P(A \cup B)$ and $P(A|B)$. 

\item If an operation on the set of integers $\mathbb{Z}$ is defined by $a*b=2a^2+b$, then find $(i)$ whether it is binary or not,  and $(ii)$ If a binary, then is it commutative or not.


\item Four cards are drawn one by one with replacement from a well-shuffeled deck of playing cards. Find the probability that atleast three cards are diamonds.

\item The Probability of two students $A$ and $B$ coming to school on time are $\frac{2}{7}$ and $\frac{4}{7}$, respectively. Assuming that the events '$A$ coming on time' and '$B$ coming on time' are independent, find the probability of only one of them coming to school on time.

\item if \begin{align*}x^py^q=(x+y)^{p+q}\end{align*} prove that $\frac{dy}{dx}=\frac{y}{x}$ and $\frac{d^2y}{dx^2}=0$.
		
\item Find \begin{align*}\int (\sin x \sin 2x \sin 3x) dx\end{align*}

\item Difference \begin{align*}\tan^{-1}\frac{3x-x^3}{1-3x^2}\end{align*}, $|x|<\frac{1}{\sqrt{3}}$ w.r.t $\tan^{-1}\frac{x}{\sqrt{1-x^2}}$

\item If \begin{align*}\sqrt{1-x^2}+\sqrt{1-y^2}=a(x-y)\end{align*}, $|x|<1,|y|<1$, show that $\frac{dy}{dx}=\sqrt{\frac{1-y^2}{1-x^2}}$

\item Find the particular solution of the differential equation :\begin{align*}x\frac{dy}{dx}\sin\frac{y}{x}+x-y\sin\frac{y}{x}=0\end{align*}, given that $y(1)=\frac{\pi}{2}$.

	\item Find the particular solution of the differential equation :\begin{align*}(1+e^{2x})\,dy+(1+y^2)e^x\,dx=0\end{align*} given that $y(0)=1$.	

\item Prove that the relation $R$ in the set $A=\{1,2,3,4,5,6,7\}$ given by $R=\{(a,b) : |a-b|$ is even$\}$ is an equivalence relation.

\item Show that the function $f$ in $A=\mathbb{R}-\{\frac{2}{3}\}$ defined as $f(x)=\frac{4x+3}{6x-4}$ is one-one and onto. Hence, find $f^{-1}$.
			
\item Find whether the function $f(x)=\cos(2x+\frac{\pi}{4})$; is increasing or decreasing in the interval $\frac{3\pi}{8}<x<\frac{5\pi}{8}$.

\item Find the equation of the plane passing through the point $(-1,3,2)$ and perpendicular to the planes $x+2y+3z=5$ and $3x+3y+z=0$.	

\item Prove that \begin{align*}\sin^{-1}\frac{4}{5}+\tan^{-1}\frac{5}{12}+\cos^{-1}\frac{63}{65}=\frac{\pi}{2}\end{align*}

\item \begin{align*}\int_{0}^{1}(|x-1|+|x-2|+|x-4|)\,dx\end{align*}

\item Using vectors find the value of x such that the four points $A(x,5,-1)$, $B(3,2,1)$ ,$C(4,5,5)$ and $D(4,2,-2)$ are coplanar.	

\item If $x,y,z$ are the different and $\triangle =\begin{vmatrix}    
		                                        x & x^2 & x^3-1\\
							y & y^2 & y^3-1\\
							z & z^2 &z^3-1\\
\end{vmatrix}=0$, then using properties of determinants , show that $xyz=1$.

\item Using integration, find the area of triangle $ABC$ bounded by the lines $4x-y+5=0$, $x+y-5=0$ and $x-4y+5=0$.

\item Using integration, find the area of the following region:\begin{align*}{(x,y): x^2+y^2\leq 16a^2} and {y^2 \leq 6ax}\end{align*}		

\item Find the vector equation of the line passing through $(2,1,-1)$ and parallel to the line $\vec{r}$=$(\vec{i}+\vec{j})+\lambda(2\vec{i}-\vec{j}+\vec{k})$. Also, find the distance between these two lines.
	
\item Find the coordinates of the foot $Q$ of the perpendicular drawn from the point $P(1,3,4)$ to the plane $2x-y+z+3=0$. Find the distance $PQ$ and the image of $P$ treating the plane as a mirror.

\item A company manufactures two types of novelty souvenirs made of plywood. Souvenirs of type $A$ require $5$ minutes each for cutting and $10$ minutes each for assembling. Souvenirs of type $B$ require $8$ minutes each for cutting and $4$ hours for assembling. There are $3$ hours and $20$ minutes available for cutting and $4$ hours for assembling. The Profit for type $A$ souvenirs is $100$ rupees each and for type $B$ souvenirs, profit is of $120$ rupees each.How many souvenirs of each type should the company manufacture in order to maximize the profit ? Formulate the problem as LPP and then solve it graphically.

\item In answering a question on a multiple choice questons with four choices in each question, out of which only one is correct, a student either guesses or copies or knows the answer. The probability that he makes a guess is $\frac{1}{4}$ and the probability that he copies is also $\frac{1}{4}$. The probability that the answer is correct, given that he copied it is $\frac{3}{4}$. Find the probability that he knows the answer to th question, given that he correctly answered it.

\item An isosceles triangle of vertical angle $2\theta$ is inscribed in a circle  of radius $a$. Show that the area of the triangle is maximum when $\theta =\frac{\pi}{6}$.

\item Using elementry row transformations, find the inverse of the matrix
		$\begin{bmatrix}
			3 & 0 &-1\\
			2 & 3 & 0\\
			0 & 4 &1\\
		\end{bmatrix}$

\item Using matrices, solve the following system of linear equations:
	$2x+3y+10z=4$,
	$4x+6y+5z=1$,
	$6x+9y-20z=2$



\end{enumerate}



\end{document}

