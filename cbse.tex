\documentclass[12pt,-letter paper]{article}                       
\usepackage{siunitx}                                              
\usepackage{setspace}
\usepackage{gensymb}                                              
\usepackage{xcolor}                                               
\usepackage{caption}
%\usepackage{subcaption}
\doublespacing                                                    
\singlespacing                                                    
\usepackage[none]{hyphenat}
\usepackage{amssymb}
\usepackage{relsize}
\usepackage[cmex10]{amsmath}
\usepackage{mathtools}
\usepackage{amsmath}                                              
\usepackage{commath}                                              
\usepackage{amsthm}
\interdisplaylinepenalty=2500
%\savesymbol{iint}\usepackage{txfonts}                                              
%\restoresymbol{TXF}{iint}                                        
\usepackage{wasysym}                                              
\usepackage{amsthm}
\usepackage{mathrsfs}                                             
\usepackage{txfonts}                                              
\let\vec\mathbf{}
\usepackage{stfloats}
\usepackage{float}
\usepackage{cite}
\usepackage{cases}                                                
\usepackage{subfig}                                               
%\usepackage{xtab}
\usepackage{longtable}
\usepackage{multirow}
%\usepackage{algorithm}
\usepackage{amssymb}
%\usepackage{algpseudocode}
\usepackage{enumitem}
\usepackage{mathtools}
%\usepackage{eenrc}
%\usepackage[framemethod=tikz]{mdframed}                          
\usepackage{listings}                                             
%\usepackage{listings}
\usepackage[latin1]{inputenc}
%%\usepackage{color}{
%%\usepackage{lscape}
\usepackage{textcomp}
\usepackage{titling}
\usepackage{hyperref}
%\usepackage{fulbigskip}
\usepackage{tikz}
\usepackage{graphicx}                                             
\lstset{
  frame=single,
  breaklines=true
}
\let\vec\mathbf{}
\usepackage{enumitem}                                             
\usepackage{graphicx}                                             
\usepackage{siunitx}
\let\vec\mathbf{}                                                 
\usepackage{enumitem}
\usepackage{graphicx}
\usepackage{enumitem}
\usepackage{tfrupee}
\usepackage{amsmath}
\usepackage{amssymb}
\usepackage{mwe} % for blindtext and example-image-a in example
\usepackage{esvect}
\usepackage{wrapfig}
\usepackage{physics}
\newcommand{\myvec}[1]{\ensuremath{\begin{pmatrix}#1\end{pmatrix}}}
\let\vec\mathbf

\newcommand{\mydet}[1]{\ensuremath{\begin{vmatrix}#1\end{vmatrix}}}
\providecommand{\brak}[1]{\ensuremath{\left(#1\right)}}
\begin{document}


\section{Vector Algebra}
\begin{enumerate}

\item  Find the angle between the line $\overrightarrow{r}$ = $\brak{2\hat{i}-\hat{j}+\hat{k}}$+ $\lambda\brak{3\hat{i}-\hat{j}+2\hat{k}}$ and the plane $\overrightarrow{r}\cdot\brak{\hat{i} +\hat{j}+\hat{k}}=3$.

\item Find the co-ordinates of the point where the line $\dfrac{x+2}{1}=\dfrac{y-5}{3}=\dfrac{z+1}{5}$ cuts the $yz$-plane.

\item Using vectors, prove that the points $\brak{2,-1,3}$, $\brak{3,-5,1}$ and $\brak{-1,11,9}$ are collinear.

\item For any two vectors $\overrightarrow{a}$ and $\overrightarrow{b}$  prove that 
	\begin{align*}
	(\overrightarrow{\vec{a}}\times\overrightarrow{\vec{b}})^{2}=\overrightarrow{\vec{a}}^{2}\overrightarrow{\vec{b}}^{2}-(\overrightarrow{\vec{a}}.\overrightarrow{\vec{b}})^{2}
	\end{align*}
 
\item Find the equation of the plane passing through the point $\brak{-1,3,2}$ and perpendicular to the planes $x+2y+3z=5$ and $3x+3y+z=0$.

\item Find the vector equation of the line passing through $\brak{2,1,-1}$ and parallel to the line $\overrightarrow{r}$=$\brak{\hat{i}+\hat{j}}+\lambda\brak{2\hat{i}-\hat{j}+\hat{k}}$. Also find the distance between these two lines.

 \item Find the equation of the plane passing through the point $(-1,3,2)$ and perpendicular to the planes $x+2y+3z=5$ and $3x+3y+z=0$.
 
\item Find the coordinates of the foot $Q$ of the perpendicular drawn from the point $P\brak{1,3,4}$ to the plane $2x-y+z+3=0$. Find the distance $PQ$ and the image of $P$ treating the plane as a mirror. 

\item Using vectors find the value of $\vec{x}$ such that the four points $A\brak{x,5,-1}$, $B\brak{3,2,1}$ ,$C\brak{4,5,5}$ and $D\brak{4,2,-2}$ are co-planar.


\end{enumerate}

\section{Matrices}

\begin{enumerate}

\item If $A$ is a square matrix of order $2$ and $\mydet{A}=4$, then find the value of $\mydet{2.AA^{\prime}}$ where $A^\prime$ is the transpose of matrix A.       
       
\item Find the value of $\brak{x-y}$ from the matrix equation 
	\begin{align*}
		2\myvec{x & 5\\7 & y-3}+\myvec{-3 & -4\\1 & 2}=\myvec{7 & 6\\15 & 14}
	\end{align*}  

\item If $x,y,z$ are the different and $\triangle =
                    \begin{vmatrix}    
		                      x & x^2 & x^3-1\\
				      y & y^2 & y^3-1\\
				      z & z^2 & z^3-1
                    \end{vmatrix}=0$  then using properties of determinants show that $xyz=1$.

\item Using elementary row transformations find the inverse of the matrix
		$\begin{bmatrix}
			3 & 0 &-1\\
			2 & 3 & 0\\
			0 & 4 &1
		\end{bmatrix}$

\item Using matrices solve the following system of linear equations:
	\begin{align*}
		    2x+3y+10z=4\\
	        4x+6y+5z=1\\
	        6x+9y-20z=2
	\end{align*}

\item Find the value of $\brak{x-y}$ from the matrix equation 
	\begin{align*}
		2\myvec{x & 5\\7 & y-3}+\myvec{-3 & -4\\1 & 2}=\myvec{7 & 6\\15 & 14}
	\end{align*}
 
\end{enumerate}

\section{Functions and Relations}

\begin{enumerate}

	\item Prove that the relation $R$ in the set $A=\{1,2,3,4,5,6,7\}$ given by $R$=$\brak{a,b} : \abs{a-b}$ is even, is an equivalence relation.

\item Show that the function $f$ in $A=R-\{\dfrac{2}{3}\}$ defined as $f(x)=\dfrac{4x+3}{6x-4}$ is one-one and onto. Hence find $f^{-1}$. 

\item Find whether the function $f(x)=\cos\brak{2x+\dfrac{\pi}{4}}$ is increasing or decreasing in the interval $\dfrac{3\pi}{8}<x<\dfrac{5\pi}{8}$.

\item If an operation on the set of integers $\mathbb{Z}$ is defined by $a*b=2a^2+b$, then find $(i)$ whether it is binary or not and $(ii)$ If a binary  then is it commutative or not.

\end{enumerate}

\section{Integrations}

\begin{enumerate}

\item Find \begin{align*}\int e^x\brak{\cos x-\sin x}\csc^2 x dx\end{align*}

\item Find \begin{align*}\int \frac{x-1}{(x-2)(x-3)}\,dx\end{align*}

\item Integrate \begin{align*}\frac{e^x}{\sqrt{5-4e^x-e^{2x}}}\end{align*} with respect to $x$.

\item Find \begin{align*}\int \brak{\sin x \sin 2x \sin 3x} dx\end{align*}

\item Evaluate \begin{align*}\int_{0}^{1}\brak{|x-1|+|x-2|+|x-4|} dx\end{align*}

\item Find :
\begin{align*}
\int{e^x\left(\frac{2+\sin 2x}{2\cos^2 x}\right) dx}
\end{align*}

\end{enumerate}

\section{Differentiation}

\begin{enumerate}

\item If \begin{align*}y= 5e^{7x}+6e^{-7x}\end{align*}  show that $\dfrac{d^2y}{dx^2}=49y$

\item if \begin{align*}x^py^q=(x+y)^{p+q}\end{align*} and prove that $\dfrac{dy}{dx}=\dfrac{y}{x}$ and $\dfrac{d^2y}{dx^2}=0$

\item Differentiate \begin{align*}\tan^{-1}\frac{3x-x^3}{1-3x^2}\end{align*}  $|x|<\dfrac{1}{\sqrt{3}}$ w.r.t $\tan^{-1}\dfrac{x}{\sqrt{1-x^2}}$

\item If \begin{align*}\sqrt{1-x^2}+\sqrt{1-y^2}=a\brak{x-y} \quad\abs{x}<1,\abs{y}<1\end{align*}  show that $\dfrac{dy}{dx}=\sqrt{\dfrac{1-y^2}{1-x^2}}$

 \item An isosceles triangle of vertical angle $2\theta$ is inscribed in a circle  of radius $a$. Show that the area of the triangle is maximum when $\theta =\dfrac{\pi}{6}$.

\end{enumerate}

\section{Differential Equations}
\begin{enumerate}

\item Find the order of differential equation of the family of circles of radius $3$ units.

\item Solve the equation differential equation \begin{align*}\brak{y+3x^2}\frac{dx}{dy}=x\,dx \end{align*}

\item Find the particular solution of the differential equation :
	\begin{align*}
             x\frac{dy}{dx}\sin\brak{\frac{y}{x}}+x-y\sin\brak{\frac{y}{x}}=0
        \end{align*} given that $y(1)=\dfrac{\pi}{2}$
        
\item Find the particular solution of the differential equation :
	\begin{align*}
	      \brak{1+e^{2x}}dy+\brak{1+y^2}e^x dx=0
	\end{align*} given that $y(0)=1$

\item Find the differential equation representing the family of curves \begin{align*}y= -A\cos 3x+B\sin 3x\end{align*}

\item Find the differential equation of the function $\cos^{-1}(\sin 2x)$ w.r.t. $x$.

\end{enumerate}

\section{Probability}

\begin{enumerate}
\item If $P\brak{A}=0.6, P\brak{B}=0.5$ and $P\brak{\dfrac{B}{A}}=0.4$ find $P\brak{A \cup B}$ and $P\brak{\dfrac{A}{B}}$.

\item Four cards are drawn one by one with replacement from a well-shuffled deck of playing cards. Find the probability that at least three cards are diamonds.

\item The Probability of two students $A$ and $B$ coming to school on time are $\dfrac{2}{7}$ and $\dfrac{4}{7}$ respectively, Assuming that the events '$A$ coming on time' and '$B$ coming on time' are independent find the probability of only one of them coming to school on time.

\item In answering a question on a multiple choice questions with four choices in each question out of which only one is correct a student either guesses or copies or knows the answer. The probability that he makes a guess is $\dfrac{1}{4}$ and the probability that he copies is also $\dfrac{1}{4}$. The probability that the answer is correct given that he copied it is $\dfrac{3}{4}$. Find the probability that he knows the answer to the question given that he correctly answered it.

\item If $A$ and $B$ are independent events with $P(A)=\dfrac{3}{7}$ and $P(B)=\dfrac{2}{5}$, then find $P(A' \cap B')$

\end{enumerate} 

\section{Trigonometric Identities}
\begin{enumerate}
    \item Prove that 
\begin{align*}
    \sin^{-1}\frac{4}{5}+\tan^{-1}\frac{5}{12}+\cos^{-1}\frac{63}{65}=\frac{\pi}{2}
\end{align*}

\end{enumerate}

\section{Linear Programming}
\begin{enumerate}

\item A company manufactures two types of novelty souvenirs made of plywood. Souvenirs of type $A$ require $5$ minutes each for cutting and $10$ minutes each for assembling. Souvenirs of type $B$ require $8$ minutes each for cutting and $4$ hours for assembling. There are $3$ hours and $20$ minutes available for cutting and $4$ hours for assembling. The Profit for type $A$ souvenirs is $100$ rupees each and for type $B$ souvenirs, profit is of $120$ rupees each. How many souvenirs of each type should the company manufacture in order to maximize the profit? Formulate the problem as LPP and then solve it graphically.

\end{enumerate}      
\end{document}
